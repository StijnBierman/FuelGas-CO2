\documentclass[17pt,a4paper,notitlepage]{article}
\pagestyle{plain}
%\usepackage[retainorgcmds]{IEEEtrantools}
\usepackage{amsmath}
\usepackage{amsthm}
\usepackage{graphicx}
%\usepackage{extsizes}
%\usepackage[font=small]{caption}\textsc{\textsc{}}
\usepackage{url}
\usepackage{natbib}
\bibliographystyle{plainnat}
\usepackage{hyperref}
\usepackage[absolute]{textpos}
\setlength{\TPHorizModule}{1mm}
\setlength{\TPVertModule}{1mm}
%\usepackage[toc,page]{appendix}

\usepackage[left=2cm,top=2cm,right=2cm]{geometry}
\input{Notation}

\begin{document}

\title{Statistical methods to estimate the mean annual Emission Factor of a fuel gas, and confidence interval of the estimated mean. }
\author{Stijn Bierman}
\date{\today}
\maketitle

\abstract{Reliable estimates of greenhouse gas emission of refineries and petrochemical plants are required in order to demonstrate both regulatory compliance and continuous improvement in terms of reductions in net emissions relative to the  amounts of produced products. For example, if refineries in The Netherlands are to take part in $\text{CO}_2$ emissions trading, the Dutch emissions authority (NEA) requires the annual average Emission Factor of a fuel gas (EF; the amount of produced $\text{CO}_2$ per amount of burnt fuel gas) to be estimated with a prescribed minimum level of precision. }
\tableofcontents  % Mandatory item, do not remove it.
%\listoffigures    % This can be remarked out if not needed.
%\listoftables     % This can be remarked out if not needed.

\clearpage
\section{Introduction}\label{Intro}

Trustworthy and precise estimates of greenhouse gas emissions of refineries and petrochemical plants are essential to measure reductions in net emissions per amount of produced product and to demonstrate regulatory compliance. For example, refineries in The Netherlands which want to take part in $\text{CO}_2$ emissions trading, are required by the Dutch Emissions Authority (NEA) to report estimates of the annual average Emission Factor (EF; the amount of produced $\text{CO}_2$ per amount of combusted fuel gas; see e.g. \cite{API2009}) of each fuel gas. For each fuel gas stream, the annual average EF can be multiplied with the annual total amount of combusted fuel gas to yield an estimate of the total amount of $\text{CO}_2$ produced. The NEA requires these estimates of annual average EF to have an associated minimum precision. 

The EF of a single sample, taken from the flow of a fuel gas, can be measured in the laboratory. The cost per sample is relatively high, and a large number of samples may be needed to estimate the annual average EF with sufficient precision. A well known and potentially highly efficient way to more precise estimates, for the same number of laboratory samples, is to use one or more auxiliary variables which correlate with laboratory measurements of EF and which can be measured at lower cost and higher frequency (see e.g. chapters 6 and 7 in \cite{Cochran77} and chapters 7 and 8 in \cite{Thompson}). Ideally, an auxiliary variable correlates strongly with EF and can be measured continuously using an online measurement device which is permanently in place. The auxiliary variables are not of direct interest themselves, but can help in obtaining estimates with higher precision.

A large body of scientific literature exists on how to estimate means or totals based on samples and auxiliary variables. Excellent overviews are given by \citet{Cochran77} and \citet{Thompson}. Few people are familiar with this literature, and there are no universally applicable rules which produce valid and efficient estimates in all situations. It is not straightforward to select an appropriate statistical procedure for a given situation, and to transparently demonstrate that the assumptions are met for the method to yield valid estimates. It seems worthwhile therefore to build up a body of knowledge on appropriate statistical methodology which may be used to obtain such estimates. 

This report provides and overview of statistical recipes (estimators) which yield estimates of the annual average EF and associated precision, based on laboratory measurements alone or on a combination of a continuously monitored auxiliary variable and laboratory measurements. The estimators are applied to synthetic data sets, generated under scenarios which capture a range of different situations which may be encountered in practice. The scenarios are defined in terms of: 
\begin{itemize}
	\item The number of laboratory measurements
	\item The correlation (strength of relationship) between the auxiliary variable and the laboratory measurements
	\item The variability in EF, auxiliary variable and flow rate measurements (due to a combination of process variability and measurement errors)
	\item The correlation (strength of relationship) between flow rate and EF, and between flow rate and auxiliary variable
\end{itemize} 

Estimators are compared in terms of their precision, measured by the width of the 95\% confidence interval of the mean, and their coverage probability, i.e. the proportion of cases in which the confidence interval contains the true average EF. If an estimator produces confidence intervals with a coverage probability (substantially) below 95\%, then the  precision estimates based on this estimator are unjustifiably high and as such potentially misleading. For each scenario, the most efficient estimator yields an estimate with the highest precision whilst maintaining an acceptable coverage probability. 

%\clearpage
\section{Summary of main Results}\label{Discussion}
A number of statistical procedures (estimators) for estimating the mean annual EF as well as the relative uncertainty of the estimate of the mean, have been described and assessed using a simulation study. The aim was to better understand the performance of these estimators, in terms of their relative uncertainties and coverage probabilities, under different scenarios for variability in EF measurements, flow rates, sample sizes and correlations between the key variables.

The main learnings are:
\begin{itemize}
	\item The use of an auxiliary variable has the potential to greatly increase the precision of the estimate of the annual average EF. In order to obtain very high precision of estimates, for example less than 0.5\% relative uncertainty, it is necessary to have an auxiliary variable which correlates strongly with EF measurements.
	\item Without an auxiliary variable, a large number of samples is required to get high precision.
	\item A simulation tool has been developed in an App that can be used to find the required sample size to obtain a specified precision, as a function of the main variables of interest: process variability in EF, variability in flow rates and correlations between the variables of interest. The parameters can be estimated from actual process data. An example from a fuel gas in the Pernis refinery is included,
	\item Estimators based on auxiliary variables are model based, so model checking is required as outlined in \citet{vanZanten} and in chapter~\ref{Validation}.
	\item The van Zanten estimator has good performance and good coverage probability, except when the variability in flow rate is high and the flow rate is correlated with EF.
	\item The coverage of the linear regression estimator is slightly too low if the variability in flow rates is high. The performance of the linear regression estimator does not deteriorate when there is a correlation between flow rate and EF.
	\item The Bootstrap estimators have good performance, and can readily be extended to multiple regression, or other types of relationships of needed.
	\item The following factors have a particularly large influence on the relative uncertainty:
	\begin{itemize}
		\item The sample size $n$ (number of laboratory measurements on samples taken from the flow of the fuel gas): the larger the sample size, the smaller the relative uncertainty.
		\item The process variability in EF $\sigma_Y$: the larger the process variability in EF the larger the relative uncertainty.
		\item The correlation between the auxiliary variable and EF $\rho_{XY}$: the stronger the correlation (the closer it is to 1) the lower the relative uncertainty.
	\end{itemize}
\end{itemize}

If there is a single auxiliary variable and there is no correlation between flow rate and EF, then the van Zanten estimator is appropriate if the other assumptions that apply to this method are reasonable (see \citet{vanZanten} and chapter~\ref{Validation}).
If, in addition to the above, the variability in flow rates is low, then the regression estimator from \citet{Cochran77} is also acceptable. This estimator has the advantage that it is described in textbooks and is easy to understand.

The Bootstrap methods also perform well, and have the advantage that they can be readily extended to other situation, in particular for multiple regression when multiple auxiliary variables are available. A disadvantage of the Bootstrap is that it can only be implemented using specialist statistical software.

An analysis of data from a fuel gas in the Pernis refinery indicates a number of potential candidates for auxiliary variables. A multiple linear regression model of EF on Stoichiometric Air Requirement (SAR) or Lower Heating Value (LHV) and molar weight yields predictions that correlate very strongly with measured EF (figure~\ref{fig:SAR_y_predicted_SAR_molweight_peryear}). This is encouraging because, if a device can be found which can measure online a variable which correlates with SAR or LHV, then this may result in a model with good predictive performance in combination with online molar weight measurements.


\clearpage
\bibliography{ref}
\clearpage


\end{document}
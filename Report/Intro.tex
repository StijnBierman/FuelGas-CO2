\clearpage
\section{Introduction}\label{Intro}

Trustworthy and precise estimates of greenhouse gas emissions of refineries and petrochemical plants are essential to measure reductions in net emissions per amount of produced product and to demonstrate regulatory compliance. For example, refineries in The Netherlands which want to take part in $\text{CO}_2$ emissions trading, are required by the Dutch Emissions Authority (NEA) to report estimates of the annual average Emission Factor (EF; the amount of produced $\text{CO}_2$ per amount of combusted fuel gas; see e.g. \cite{API2009}) of each fuel gas. For each fuel gas stream, the annual average EF can be multiplied with the annual total amount of combusted fuel gas to yield an estimate of the total amount of $\text{CO}_2$ produced. The NEA requires these estimates of annual average EF to have an associated minimum precision. 

The EF of a single sample, taken from the flow of a fuel gas, can be measured in the laboratory. The cost per sample is relatively high, and a large number of samples may be needed to estimate the annual average EF with sufficient precision. A well known and potentially highly efficient way to more precise estimates, for the same number of laboratory samples, is to use one or more auxiliary variables which correlate with laboratory measurements of EF and which can be measured at lower cost and higher frequency (see e.g. chapters 6 and 7 in \cite{Cochran77} and chapters 7 and 8 in \cite{Thompson}). Ideally, an auxiliary variable correlates strongly with EF and can be measured continuously using an online measurement device which is permanently in place. The auxiliary variables are not of direct interest themselves, but can help in obtaining estimates with higher precision.

A large body of scientific literature exists on how to estimate means or totals based on samples and auxiliary variables. Excellent overviews are given by \citet{Cochran77} and \citet{Thompson}. Few people are familiar with this literature, and there are no universally applicable rules which produce valid and efficient estimates in all situations. It is not straightforward to select an appropriate statistical procedure for a given situation, and to transparently demonstrate that the assumptions are met for the method to yield valid estimates. It seems worthwhile therefore to build up a body of knowledge on appropriate statistical methodology which may be used to obtain such estimates. 

This report provides and overview of statistical recipes (estimators) which yield estimates of the annual average EF and associated precision, based on laboratory measurements alone or on a combination of a continuously monitored auxiliary variable and laboratory measurements. The estimators are applied to synthetic data sets, generated under scenarios which capture a range of different situations which may be encountered in practice. The scenarios are defined in terms of: 
\begin{itemize}
	\item The number of laboratory measurements
	\item The correlation (strength of relationship) between the auxiliary variable and the laboratory measurements
	\item The variability in EF, auxiliary variable and flow rate measurements (due to a combination of process variability and measurement errors)
	\item The correlation (strength of relationship) between flow rate and EF, and between flow rate and auxiliary variable
\end{itemize} 

Estimators are compared in terms of their precision, measured by the width of the 95\% confidence interval of the mean, and their coverage probability, i.e. the proportion of cases in which the confidence interval contains the true average EF. If an estimator produces confidence intervals with a coverage probability (substantially) below 95\%, then the  precision estimates based on this estimator are unjustifiably high and as such potentially misleading. For each scenario, the most efficient estimator yields an estimate with the highest precision whilst maintaining an acceptable coverage probability. 
